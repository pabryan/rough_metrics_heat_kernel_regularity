\documentclass{amsart}

\input{StandardPaper2.tex}
\DeclareMathOperator{\met}{Met}
\DeclareMathOperator{\rough}{\mathcal{R}}
\DeclareMathOperator{\domain}{\mathcal{D}}
\newcommand{\ip}[2]{\ensuremath{\langle{#1},{#2}\rangle}}

% Others: 
\newcommand{\Sum}[2]{\Sigma_{#1}^{#2}}	% Sum

% Logic definitions:
\newcommand{\disjunct}{\vee}
\newcommand{\conjunct}{\wedge}

% Mathematical niceities...
\newcommand{\mdot}{\cdotp}
\newcommand{\inte}[2]{\int_{#1}^{#2}}
\newcommand{\intbox}[3]{\left[#1\right]_{#2}^{#3}}
\newcommand{\cbrac}[1]{\left(#1\right)}
\newcommand{\bbrac}[1]{\left[#1\right]}
\newcommand{\dbrac}[1]{\left\{#1\right\}}
\newcommand{\modulus}[1]{|#1|}
\newcommand{\lmodulus}[1]{\left|#1\right|}
\newcommand{\set}[1]{\dbrac{#1}}
\newcommand{\sentence}[1]{ {\text{``}}#1 {\text{''}}}

\newcommand{\pw}[1]{\script{P}\cbrac{#1}}
\newcommand{\dom}{ {\mathcal{D}}}
\newcommand{\ran}{ {\mathcal{R}}}
\newcommand{\nul}{ {\mathcal{N}}}
%\newcommand{\im}[1]{ {\rm im} \cbrac{#1}}
%\newcommand{\card}[1]{ {\rm card} \cbrac{#1}}
\DeclareMathOperator{\card}{card}

\newcommand{\comp}{\circ}


\newcommand{\In}{\mathbb{Z}}
\newcommand{\Na}{\ensuremath{\mathbb{N}}}

\newcommand{\putt}{\ensuremath{\rightarrow}}
\newcommand{\script}[1]{\mathscr{#1}}

\newcommand{\Ccut}{\C_{\frac{\pi}{2}}}			% Cut plane
\DeclareMathOperator{\re}{Re}			% Real part
\DeclareMathOperator{\im}{Im}			% Imaginary part

\newcommand{\ceil}[1]{\ulcorner{#1}\urcorner}

% Set Theory:
\renewcommand{\emptyset}{\varnothing}
\newcommand{\union}{\cup}
\newcommand{\Union}{\bigcup}
\newcommand{\intersect}{\cap}
\newcommand{\Intersect}{\bigcap}
\newcommand{\disunion}{\sqcup}
\newcommand{\Disunion}{\bigsqcup}
%\renewcommand{\subset}{\subseteq}
\newcommand{\symdiff}{\vartriangle}
\newcommand{\rest}[1]{{{\lvert_{}}_{}}_{#1}}
\newcommand{\close}[1]{\overline{#1}}		% closure
\DeclareMathOperator{\img}{Img}
%\newcommand{\img}[1]{{\rm Img}{\text{ }#1}}	% Image

\newcommand{\powerset}{\script{P}} 

\newcommand{\ind}[1]{\raisebox{\depth}{\(\chi\)}_{#1}}	% Indicator funciton

\newcommand{\class}[1]{[#1]}			% Equivalence class
\newcommand{\quotient}[2]{{#1}/{#2}}	% Quotienting

\newcommand{\compl}{{}^{\rm c}}

\newcommand{\Char}[1]{\chi_{#1}} 	% Characteristic function 

\renewcommand{\epsilon}{\varepsilon}
\newcommand{\vphi}{\tilde{\varphi}}
\renewcommand{\phi}{\varphi}
\newcommand{\comspace}{\text{      }}
%\newcommand{\comment}[1]{\hspace{1cm}\text{\it\small(#1)}}
\DeclareMathOperator{\Graph}{graph}

% Topology:

\newcommand{\embed}{\hookrightarrow}		% Embeded in

% Algebra:
\newcommand{\tp}[1]{{#1}^{\mathrm{tr}}}
%\newcommand{\dual}[1]{#1^{\ast}}		% Algebraic Dual
\newcommand{\isomorphic}{\cong}			% Isomorphic
\newcommand{\Alt}{\mathcal{A}}
\newcommand{\tensor}{\otimes}
\newcommand{\dsum}{\oplus}			% direct sum
\DeclareMathOperator{\Span}{span}
\newcommand{\longto}{\longrightarrow}		% --->
\newcommand{\comm}[1]{\bbrac{#1}}		% Commutator
\newcommand{\trans}[1]{#1^{{\rm T}}}

% Specific Analysis things:
\newcommand{\sups}[1]{{\rm sup}\set{#1}} 	% Supremum
\newcommand{\infs}[1]{{\rm inf} \set{#1}}	% Infimum

\newcommand{\upto}{\nearrow}			
\newcommand{\downto}{\searrow}

\newcommand{\norm}[1]{\| #1 \|}			% Norm
\newcommand{\lnorm}[1]{\left\| #1 \right\|}			% Norm
\newcommand{\spt}[1]{{\rm spt} {\text{ }}#1}	% Support 
\DeclareMathOperator{\esssup}{esssup}
\DeclareMathOperator{\essinf}{essinf}
%\newcommand{\esssup}{{\rm ess {\text{ }}sup} {\text{ }} }	% Essential Supremum
\newcommand{\conv}{\rightarrow}
\newcommand{\weakcon}{\rightharpoonup}		% Weak Convergence
\newcommand{\interior}[1]{\mathring{#1}}	% Interior


% Specific GMT/Geometry things:
\DeclareMathOperator{\tr}{tr}			% Trace
\newcommand{\Secondff}{\mathbb{I}}		% 2nd FF
\newcommand{\Mean}{\vec{\mathrm{H}}}
\DeclareMathOperator{\diam}{diam}		% Diam 

\DeclareMathOperator{\len}{\ell}			% Length
\DeclareMathOperator{\rad}{rad}			% Rad
\DeclareMathOperator{\vol}{vol}			% Volume

\DeclareMathOperator{\divv}{div}		% Divergence

\newcommand{\kron}{\delta}			% Kronecker delta

\newcommand{\cut}{\ \llcorner\ }			% cut product

\newcommand{\partt}[1][{}]{{\partial_{{#1}}}}		% Partial

\newcommand{\ddt}[1][t]{\frac{d}{d#1}}		% d/dt 
\newcommand{\partd}[2][{}]{{\frac{\partial #1}{\partial #2}}}	% d/dx
\newcommand{\dx}[1]{{\rm d}#1} 

\newcommand{\Lie}[1]{\script{L}_{#1}}		% Lie Derivative
\newcommand{\VecF}{\script{X}}			% Vector Fields X
\newcommand{\Christ}[2]{\Gamma^{#2}_{#1}}	% Cristoffel Symbols

\newcommand{\Ric}{{\rm Ric}}			% Ricci Curvature
\newcommand{\Rs}{\mathcal{R}_S}			% Scalar Curvature


\newcommand{\bnd}{\partial}			% Boundary partial d
\newcommand{\Vectors}[1]{\edge_{#1}}		% P-vectors
\newcommand{\WForms}[1][{}]{\wedge^{#1}}		% P-Forms\\
\newcommand{\Forms}[1][{}]{\mathbf{\Omega}^{#1}}		% P-Forms\\
\newcommand{\Tensors}[1][{}]{{\mathcal{T}}^{(#1)}}	% Tensors
\DeclareMathOperator{\Sym}{Sym}
\newcommand{\Sect}{\mathbf{\Gamma}}		% Sections (tensor feilds)
\newcommand{\current}[1]{\script{D}^{#1}}	% 
\newcommand{\Current}[1]{\script{D}_{#1}}	%
\newcommand{\Varifold}[1]{{\rm V}\cbrac{#1}}	% Varifold

\DeclareMathOperator{\proj}{\mathbf{P}}	% Projection
\newcommand{\tanb}{{\rm T}}		% Tangent Bundle
\newcommand{\cotanb}{{\rm T}^\ast}	% Cotangent bundle
\newcommand{\normalb}{{\rm N}}		% Normal bundle
\newcommand{\conormalb}{{\rm N}^\ast}	% Conormal bundle

\newcommand{\pushf}[1]{{#1}_\ast}			% Push forward
\newcommand{\pullb}[1]{{#1}^\ast}			% Pull back

\newcommand{\metric}[1]{\inprod{#1}}

\DeclareFontFamily{OT1}{restrictfont}{}
\DeclareFontShape{OT1}{restrictfont}{m}{n}{<-> fmvr8x}{}
\def\restrictfont{\usefont{OT1}{restrictfont}{m}{n}}
\newcommand{\restrict}{\text{\restrictfont\char157}}		% Restriction L

\newcommand{\cur}[1]{\llbracket #1 \rrbracket}	% [[M]]
\newcommand{\slice}[1]{\inprod{#1}}		% Slice <T,f,t>

\newcommand{\adj}[1]{{#1}^\ast}			% Adjoint star
\newcommand{\dirD}[1]{{\rm D}_{#1}}		% Directional Deriv
\newcommand{\extd}{{\rm d}}			% Exterior Derivative
\newcommand{\intd}{{\updelta}}
\newcommand{\Dir}{{\rm D}}			% Dirac
\newcommand{\diff}[1]{{\rm d}^{#1}}		% differential d
\newcommand{\jacob}[1]{{\rm J}_{#1}}		% jacobian
\newcommand{\cojacob}[1]{\adj{{\rm J}_{#1}}}	% cojacobian
\newcommand{\integral}[3]{\int_{#1}#2  {\text{   }} {\rm d} #3}  % integral
\newcommand{\cprod}{\mathaccent\openJoin{\mathaccent\opentimes\opentimes}}		% Product of Currents

\newcommand{\inprod}[1]{\langle #1 \rangle}	% inner product braces
\newcommand{\conn}[1][{}]{{\grad_{{#1}}}}		% Connection
\newcommand{\Ae}{{\rm -a.e.~}}			% Almost everywhere
\DeclareMathOperator{\Lip}{\bf Lip}			% Lip
\DeclareMathOperator{\Lipp}{Lip}		% Lip f(x)
\newcommand{\Vol}[1]{{\rm Vol}\cbrac{#1}}	% Volume
%\newcommand{\diam}[1]{{\rm diam}\cbrac{#1}}	% Diam
\newcommand{\Haus}[1]{\script{H}^{#1}}			% Hausdorff H
\newcommand{\Leb}[1][{}]{\script{L}^{#1}}			% Lebesgue L
\newcommand{\mass}[1]{{\rm M}_{#1}}

\newcommand{\udensity}[1]{\Theta^{\ast#1}}		% Upper density
\newcommand{\ldensity}[1]{\Theta_{\ast}^{#1}}	% Lower density
\newcommand{\density}[1]{\Theta^{#1}}		% Density 

% Functional Analysis/ Operator Theory

\DeclareMathOperator{\diag}{diag}
\newcommand{\bddlf}{\mathcal{L}} 	% Bounded Linear Functions over #1
\newcommand{\spec}{\sigma}		% Spectrum of #1
\newcommand{\jspec}{\gamma}				% Joint spectrum
\newcommand{\rs}[1]{\textsf{R}_{#1}}			% Resolvent operator
\newcommand{\rset}{\rho}				% Resolvent Set
\newcommand{\polalg}{\script{P}}			% Polynomial algebra, power series algebra
\newcommand{\ratalg}[1]{\script{R}_{#1}}		% Rational algebra with no poles in #1
\newcommand{\powalg}[1]{\script{P}_{#1}}		% Power series algebra with radius larger than #1
\newcommand{\slim}{\mathrm{s}\text{-}\lim}		% Strong convegence
\newcommand{\cllf}[1]{\mathcal{C}\left(#1\right)}	% Closed Linear operators over #1
\newcommand{\hol}{\mathcal{H}}				% Holomorphic functions
\newcommand{\conj}[1]{\overline{#1}}				% complex conjugate
\DeclareMathOperator{\nr}{nr}				% Numerical Range
\DeclareMathOperator{\specr}{\spec_{\mathrm{r}}}	% Spectral radius

\newcommand{\Lp}[2][{}]{{\rm L}^{#2}_{\rm #1}}		% L_{#1}^{#2}
\newcommand{\Ck}[2][{}]{{\rm C}^{#2}_{\rm #1}}		% C^k_c
\newcommand{\Lips}[1][{}]{{\rm Lip}_{\rm #1}}		% C^k_c
\newcommand{\Sob}[2][{}]{{\rm W}^{#2}_{\rm #1}}		% W^{k,p}_{c} Sobolev space 
\newcommand{\SobH}[2][{}]{{\Sob[#1]{#2,2}}}	% H^k_{c}  Sobolev space 

\newcommand{\convolve}{\, \ast\, }

\newcommand{\ddelta}{\updelta}

\newcommand{\Sec}[1]{\mathrm{S}_{#1}}
\newcommand{\OSec}[1]{\mathrm{S}^\mathrm{o}_{#1}}


% Harmonic ANalysis

\newcommand{\mean}[1]{{\rm m}_{#1}}			% Mean.
\newcommand{\Max}{{\mathcal{M}}}			% Maximal function
\newcommand{\Cone}{\Gamma}				% Cone over a point
\newcommand{\Tent}{\mathrm{T}}					% Tent


% Script
\newcommand{\sE}{\script{E}}

% Heat kernel 
\newcommand{\hk}{\rho}

% SGN
\DeclareMathOperator{\sgn}{sgn}


\begin{document}

\title[Heat Kernel Regularity]
 {Heat Kernel Regularity For Rough Metrics}

\curraddr{}
\email{}
\date{\today}

\dedicatory{}
\subjclass[2010]{}
\keywords{}

\begin{abstract}
On smooth manifolds equipped with rough metrics, by employing a local Harnack inequality, we show there exists a heat kernel that is locally $C^{\alpha}$ regular.
\end{abstract}

\maketitle

\section{Introduction}
\label{sec:intro}

Let \(M\) be a smooth manifold. We consider \emph{rough metrics} on \(M\), the set of which we denote \(\met_{\rough} (M)\). These are metrics on \(M\) with measurable coefficients that are locally comparable to smooth smooth metrics (see \Cref{sec:rough_metrics}).

Our main theorem is the following:

\begin{thm}
On a smooth manifold \(M\), equipped with a rough metric \(g\), there exists a heat kernel \(\rho_t = \rho_t(g)\), such that
\begin{enumerate}
\item \(\rho_t > 0\) for \(t > 0\),
\item on a parabolic cylinder \(Q = K \times [t_1, t_2]\) with \(0 < t_1 < t_2\), \(K \subseteq M\) is compact, there exists an \(\alpha = \alpha(Q)\) so that \(\rho_t \in C^{\alpha}(K \times K)\) for \(t_1 \leq t \leq t_2\).
\end{enumerate}
\end{thm}

\section{Rough Metrics}
\label{sec:rough_metrics}

\begin{defn}
Let \(M\) be a smooth manifold. A set \(E \subseteq M\) is said to be \emph{Lebesgue measurable} if for all charts \(\varphi : U \subseteq M \to V \subseteq \R^n\), the set \(\varphi(E \cap U)\) is a Lebesgue measurable set of \(\R^n\).
\end{defn}

\begin{rem}
The Lebesgue measurable sets form a \(\sigma\)-algebra. The notion of Lebesgue measurable set does not involve a measure on \(M\). However, if \(\omega\) is volume form on \(M\), then it defines a (outer) measure \(E \mapsto \int_E \omega\) for which the \(\sigma\)-algebra of measurable sets is precisely the \(\sigma\)-algebra of Lebesgue measurable sets since with respect to any chart, \(\omega\) is absolutely continuous with respect to Lebesgue measure on \(\R^n\).
\end{rem}

\begin{defn}
Let \(M, N\) be smooth manifolds. A function \(f : M \to N\) is \emph{Borel measurable} if for all Borel sets \(E \subset N\), \(f^{-1} (E)\) is a Borel set of \(M\). A function \(f : M \to N\) is \emph{Lebesgue measurable} if for all Lebesgue measurable sets \(E \subset N\), \(f^{-1} (E)\) is a Lebesgue measurable set of \(M\).
\end{defn}

\begin{rem}
Equivalently, \(f\) is Lebesgue measurable if for all charts \(\phi : U \subseteq M \to V \subseteq \R^n\) and \(\psi: W \subseteq N \to Z \subseteq \R^m\), the function \(\psi \circ f \circ \phi^{-1} : \phi \circ f^{-1} (W) \subseteq \R^n \to Z \subseteq \R^m\) is Lebesgue measurable.
\end{rem}

\textbf{What about \(f^{-1}(E)\) is Lebesgue measurable for every Borel set \(E \subseteq N\)?}

\begin{defn}
A \emph{rough} metric \(g\) is a Lebesgue measurable section of \(\met(M) = (T^{\ast} M \odot T^{\ast} M)_+\), the bundle of positive definite, symmetric bilinear forms on \(TM\) that is \emph{locally comparable} to smooth metrics. That is, \(g : M \to \met(M)\) is a Lebesgue measurable function such that \(\pi \circ g = \id_M\) where \(\pi: \met(M) \to M\) is the bundle projection. To say that \(g\) is locally comparable to smooth metrics is to say that there is an open cover \(\lbrace U_{\alpha} \rbrace\) of \(M\), smooth metrics \(g_{\alpha} \in \met(U_{\alpha})\) and real constants \(C_{\alpha} > 0\) such that
\[
\frac{1}{C_{\alpha}} g_{\alpha} (X, Y) \leq g(X, Y) \leq C_{\alpha} g_{\alpha} (X, Y)
\]
for all tangent vectors \(X, Y \in TM\).
\end{defn}

\begin{rem}
By employing a partition of unity argument, we can patch together the metrics \(g_{\alpha}\) to produce a globally defined metric \(h\). If the constants \(C_{\alpha}\) are uniformly bounded above and away from zero, then our rough metric will be globally comparable to \(h\). This is automatic whenever \(M\) is compact. If \(M\) is not compact, then this need not be true. Previously a Harnack inequality and heat kernel estimates have been obtained when the metric is globally comparable to a complete metric with bounded below curvature \cite{Saloff-Coste:/1992b}. Here we do not require the assumptions of global comparability.
\end{rem}

Given a rough metric, we may locally define a \(g_{\alpha}\)-self-adjoint, measurable operator \(A_{\alpha} : U_{\alpha} \to T^1_1 U_{\alpha}\) by
\[
g(X, Y) = g_{\alpha} (A_{\alpha} (X), Y).
\]

\section{The Harnack Inequality}
\label{sec:harnack}

\begin{itemize}
\item One approach is to work in a coordinate chart, where the metric is equivalent to a smooth one. Then the heat equation becomes a divergence form parabolic equation with measureable coefficients. Compare with the work of Salof-Coste here.
\item Another approach is to try to use functional analysis techniques applied to positive operators and cones to prove the Harnack. Compare with the work of Mondino on RCD spaces as well as the work of Hamilton and Wilking for the positive cone stuff.
\end{itemize}

\end{document}
