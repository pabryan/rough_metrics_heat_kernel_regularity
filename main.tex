\documentclass[a4paper, 12pt]{amsart}
%\usepackage{showkeys}

%\usepackage{etoolbox}
%\makeatletter
%\let\ams@starttoc\@starttoc
%\makeatother
%\makeatletter
%\let\@starttoc\ams@starttoc
%\patchcmd{\@starttoc}{\makeatletter}{\makeatletter\parskip\z@}{}{}
%\makeatother

%\usepackage[parfill]{parskip}
\usepackage{vmargin}
\usepackage[colorlinks=true,linkcolor=blue,citecolor=blue,urlcolor=blue]{hyperref}
\usepackage{bookmark}
\usepackage{amsthm,thmtools,amssymb,amsmath,amscd,amsfonts}
\usepackage{mathrsfs}
\usepackage{stmaryrd}


\usepackage[bibstyle=alphabetic,citestyle=alphabetic,backend=bibtex]{biblatex}
\bibliography{Bibliography}

\usepackage{fancyhdr}
\usepackage{esint}

\usepackage{enumerate}

\usepackage{pictexwd,dcpic}

\usepackage{graphicx}
\usepackage[utf8]{inputenc}

\declaretheorem[name=Theorem,numberwithin=section]{thm}
\declaretheorem[name=Remark,style=remark,sibling=thm]{rem}
\declaretheorem[name=Lemma,sibling=thm]{lemma}
\declaretheorem[name=Proposition,sibling=thm]{prop}
\declaretheorem[name=Definition,style=definition,sibling=thm]{defn}
\declaretheorem[name=Corollary,sibling=thm]{cor}
\declaretheorem[name=Assumption,style=remark,sibling=thm]{ass}
\declaretheorem[name=Example,style=remark,sibling=thm]{example}


\numberwithin{equation}{section}

\usepackage{cleveref}
\crefname{lemma}{Lemma}{Lemmata}
\crefname{prop}{Proposition}{Propositions}
\crefname{thm}{Theorem}{Theorems}
\crefname{cor}{Corollary}{Corollaries}
\crefname{defn}{Definition}{Definitions}
\crefname{example}{Example}{Examples}
\crefname{rem}{Remark}{Remarks}
\crefname{ass}{Assumption}{Assumptions}
\crefname{not}{Notation}{Notation}
\crefname{section}{Section}{Sections}

%Symbols
\renewcommand{\~}{\tilde}
\renewcommand{\-}{\bar}
\newcommand{\bs}{\backslash}
\newcommand{\cn}{\colon}
\newcommand{\sub}{\subset}

\newcommand{\N}{\mathbb{N}}
\newcommand{\R}{\mathbb{R}}
\newcommand{\Z}{\mathbb{Z}}
\renewcommand{\S}{\mathbb{S}}
\renewcommand{\H}{\mathbb{H}}
\newcommand{\C}{\mathbb{C}}
\newcommand{\K}{\mathbb{K}}
\newcommand{\Di}{\mathbb{D}}
\newcommand{\B}{\mathbb{B}}
\newcommand{\8}{\infty}

%Greek letters
\renewcommand{\a}{\alpha}
\renewcommand{\b}{\beta}
\newcommand{\g}{\gamma}
\renewcommand{\d}{\delta}
\newcommand{\e}{\epsilon}
\renewcommand{\k}{\kappa}
\renewcommand{\l}{\lambda}
\renewcommand{\o}{\omega}
\renewcommand{\t}{\theta}
\newcommand{\s}{\sigma}
\newcommand{\p}{\varphi}
\newcommand{\z}{\zeta}
\newcommand{\vt}{\vartheta}
\renewcommand{\O}{\Omega}
\newcommand{\D}{\Delta}
\newcommand{\G}{\Gamma}
\newcommand{\T}{\Theta}
\renewcommand{\L}{\Lambda}

%Mathcal Letters
\newcommand{\cL}{\mathcal{L}}
\newcommand{\cT}{\mathcal{T}}
\newcommand{\cA}{\mathcal{A}}
\newcommand{\cW}{\mathcal{W}}

%Mathematical operators
\newcommand{\INT}{\int_{\O}}
\newcommand{\DINT}{\int_{\d\O}}
\newcommand{\Int}{\int_{-\infty}^{\infty}}
\newcommand{\del}{\partial}

\newcommand{\inpr}[2]{\left\langle #1,#2 \right\rangle}
\newcommand{\fr}[2]{\frac{#1}{#2}}
\newcommand{\x}{\times}
\DeclareMathOperator{\Tr}{Tr}
\DeclareMathOperator{\Id}{Id}

\DeclareMathOperator{\dive}{div}
\DeclareMathOperator{\id}{id}
\DeclareMathOperator{\pr}{pr}
\DeclareMathOperator{\Diff}{Diff}
\DeclareMathOperator{\supp}{supp}
\DeclareMathOperator{\graph}{graph}
\DeclareMathOperator{\osc}{osc}
\DeclareMathOperator{\const}{const}
\DeclareMathOperator{\dist}{dist}
\DeclareMathOperator{\loc}{loc}
\DeclareMathOperator{\grad}{grad}
\DeclareMathOperator{\ric}{Ric}
\DeclareMathOperator{\Rm}{Rm}
\DeclareMathOperator{\weingarten}{\mathcal{W}}
\DeclareMathOperator{\inj}{inj}

%Environments
\newcommand{\Theo}[3]{\begin{#1}\label{#2} #3 \end{#1}}
\newcommand{\pf}[1]{\begin{proof} #1 \end{proof}}
\newcommand{\eq}[1]{\begin{equation}\begin{alignedat}{2} #1 \end{alignedat}\end{equation}}
\newcommand{\IntEq}[4]{#1&#2#3	 &\quad &\text{in}~#4,}
\newcommand{\BEq}[4]{#1&#2#3	 &\quad &\text{on}~#4}
\newcommand{\br}[1]{\left(#1\right)}

%Logical symbols
\newcommand{\Ra}{\Rightarrow}
\newcommand{\ra}{\rightarrow}
\newcommand{\hra}{\hookrightarrow}
\newcommand{\mt}{\mapsto}

%Names
\newcommand{\holder}{H\"older}

%Fonts
\newcommand{\mc}{\mathcal}
\renewcommand{\it}{\textit}
\newcommand{\mrm}{\mathrm}

%Spacing
\newcommand{\hp}{\hphantom}


%\parindent 0 pt

\protected\def\ignorethis#1\endignorethis{}
\let\endignorethis\relax
\def\TOCstop{\addtocontents{toc}{\ignorethis}}
\def\TOCstart{\addtocontents{toc}{\endignorethis}}


\newcommand{\note}[1]{\Rd {\bf[[ #1 ]]} \Bk}

\DeclareMathOperator{\met}{Met}
\DeclareMathOperator{\rough}{\mathcal{R}}
\DeclareMathOperator{\domain}{\mathcal{D}}
\newcommand{\ip}[2]{\ensuremath{\langle{#1},{#2}\rangle}}

\newcommand{\Bk}{\color{black}}
\newcommand{\Pu}{\color{purple}}
\newcommand{\Rd}{\color{red}}
\newcommand{\Bl}{\color{blue}}


\begin{document}

\title{Heat kernels and regularity for rough metrics}

\author{Lashi Bandara}
\author{Paul Bryan}

\curraddr{}
\email{}
\date{\today}

\dedicatory{}
\subjclass[2010]{}
\keywords{}

\parindent0cm
\setlength{\parskip}{\baselineskip}


\begin{abstract}
We consider smooth manifolds equipped with rough metrics
we demonstrate the existence of globally continuous
heat kernels that are locally Hölder. This is done 
by employing local Harnack estimate for weak
solutions of divergence form elliptic equations
with bounded measurable coefficients in weighted Sobolev
spaces.
\end{abstract}

\maketitle

\section{Introduction}
\label{sec:intro}
The existence and regularity of heat kernels on smooth manifolds, with smooth (or even continuous)
metrics is now a classical fact (c.f. \cite{}). However, there has been
little in the way of investigation
of this topic for metrics beneath continuous coefficients. 

A wide and useful class of potentially discontinuous metrics are called \emph{rough metrics}, 
which are guaranteed to have only measurable coefficients, and they are Riemannian-like in the sense
that they are locally bounded above and below almost-everywhere by a smooth metric.
These metrics became of interest as geometric invariances of the Kato square
root problem, particularly in the non-compact setting, in \cite{BMc, BRough}. 
In the compact case, these metrics were used to study regularity properties
of a geometric flow, weakly tangential to the Ricci flow, 
in \cite{BLM, BCont}. 

Since the manifolds we consider are smooth (topologically), we always have an 
exterior derivative. Moreover, rough metrics, like their classical continuous
or smooth counterparts, induce a canonical volume via the 
expression $d\mu(x) = \sqrt{\det g(x)} dx$ for rough metrics $g$.
Moreover, the exterior derivative is a closable object
in $\Lp{2}(M)$, which allows us to construct Sobolev spaces $\SobH{1}(M)$ 
and $\SobH[0]{1}(M)$. Here the  
former space is the  energy for the Neumann Laplacian
and the latter for the Dirichlet counterpart.
With respect to these Laplacians, we can now ask
whether a heat kernel exists, and how regular one
can expect such an object to be. In the setting of rough metrics, our main theorem
is the following. 

\begin{thm}
On a smooth manifold \(M\), equipped with a rough metric \(g\), there exists a 
Dirichlet heat kernel \(\hk_t^g = \hk_t\), such that
\begin{enumerate}
\item \(\hk_t > 0\) for \(t > 0\),
\item on a parabolic cylinder \(Q = K \times [t_1, t_2]\) with \(0 < t_1 < t_2\), \(K \subseteq M\) is compact, there exists an \(\alpha = \alpha(Q)\) so that \(\hk_t^g \in C^{\alpha}(K \times K)\) for \(t_1 \leq t \leq t_2\).
\end{enumerate}
\end{thm}


Note that heat kernels have been considered
for some time in more general settings than smooth Riemannian manifolds,
with the bulk of the literature focusing on metric spaces with bounds
on certain synthetic notions of curvature.
Many classical results can be recovered from this general 
theory since smooth (or continuous) metrics induce an associated intrinsic
distance structure. A priori, there is no reason
why a rough metric should induce a canonical distance structure - at least
this cannot be done via minimisation over curves since the
length functional may not be a well-defined device. 

The significance of our results in this paper is to demonstrate
the construction of heat kernels in this setting where we may 
not have a canonical distance but instead, we have well-defined Laplacians. 
Unlike in many classical treatments which construct the \emph{minimal} 
heat kernel via local-to-global methods, our construction is purely global.
It is likely that in many instances we are sacrificing minimality but if we fix
an energy space, we are obtaining an object that is unique to that energy.

The former set of methods typically yield a so-called minimal heat kernel,
but additional assumptions, such as a lower bound on Ricci curvature,
is assumed in order to obtain its uniqueness (see \cite{Chavel}).
More seriously, as \cite{Someone?} explains, the following Varadahn's formula
$$ d^2(x,y) = 4 \lim_{t \to 0} t \log \hk_t(x,y)$$
may fail for this minimal heat kernel. 
This formula is a strong motivating factor for our paper: 
in the rough setting, we do not have a canonical notion of
distance, but given that we have a canonical Laplacian, 
any distance we may attempt to construct for a rough 
metric should at least allow us to establish this formula.
In the smooth setting, we can look at the heat equation
in $\Lp{2}$, and it is well known that
this formula holds for the heat kernel corresponding to Dirichlet Laplacian (i.e., 
with energy in $\SobH[0]{1}(M)$). 

The method of constructing the global heat kernel
via Riesz representation methods is made known to us by
the book of Davies in \cite{Davies}.
In \cite{BCont} where the manifold was assumed to be compact, 
the existence and regularity of the heat kernel was
reduced to \emph{Harnack estimates}, which were
furnished by observing that  
the rough metric was globally comparable to a smooth metric
due to the compactness of the underlying manifold and
through the  results of Sallof-Coste in \cite{SC}.
In our situation, these results do not apply without additional assumptions
and so instead, we demonstrate
how to obtain the heat kernel via \emph{local Harnack estimates}. 

\section*{Acknowledgements}

Kaj Nyström, KAW, Warwick, Lashi's parents? 

\section{Rough Metrics}
\label{sec:rough_metrics}

The term ``rough metric'' was coined in \cite{BRough} as they were 
recognised to be the geometric invariances of the Kato square root problem.
We emphasise here that similar notions existed implicitly in the 
literature with the work of Norris in \cite{Norris} 
and \cite{SC} by Saloff-Coste.
The treatise of rough metrics we give here is rudimentary and 
a more detailed exposition is available in section 3 in \cite{BRough}. 
We begin with the following definition, which recognises
that a manifold affords us not only with a metric independent 
differentiable structure, but also a metric independent measure
structure.

\begin{defn}
Let \(M\) be a smooth manifold. A set \(E \subseteq M\) is said to be \emph{Lebesgue measurable} if for all charts \(\varphi : U \subseteq M \to V \subseteq \R^n\), the set \(\varphi(E \cap U)\) is a Lebesgue measurable set of \(\R^n\).
\end{defn}

\begin{rem}
The Lebesgue measurable sets form a \(\sigma\)-algebra. The notion of Lebesgue measurable set does not involve a measure on \(M\). 
However, if \(\omega\) is volume \Rd measure \Bk on \(M\) associated to a smooth or continuous metric, then it defines a (outer) measure \(E \mapsto \int_E\ d\omega\) for which the \(\sigma\)-algebra of measurable sets is precisely the \(\sigma\)-algebra of Lebesgue measurable sets.
A strong motivating factor for this paper is to set
constraints 
\Rd This can be seen by considering charts that are compact and observing that \(\omega\) is absolutely continuous with respect to the pullback of the Lebesgue measure inside that chart. \Bk
\end{rem}

Given that we have a metric independent measure structure, we can say that a function
$f: M \to \R$ is measurable if $f^(-\infty, \alpha]$ is measurable for every $\alpha  \in \R$.
Clearly, this extends to functions $f:M \to \C$ and on letting
 $T^{p,q}M$ denote the bundle of $(p,q)$ tensors of covariant rank $p$ and contravariant rank $q$,
we can consider measurable sections and we denote the set of such 
sections by $\Gamma(T^{p,q}M)$. 

\begin{defn}[Rough metric]
A \emph{rough} metric \(g\) is a Lebesgue measurable section of \(\met(M) = (T^{\ast} M \odot T^{\ast} M)_+\), the bundle of positive definite, symmetric bilinear forms on \(TM\) that is \emph{locally comparable}: 
for each $x \in M$, there is a chart $\psi_x:V_x \to \R^n$ and a constant $C_x = C_x(V_x)$ such that
\[
\frac{1}{C_{x}} \pullb{\psi_x}\delta_{\R^n}(y) (X, Y) \leq g(y)(X, Y) \leq C_{x} \pullb{\psi_x}\delta_{\R^n}(y) (X, Y)
\]
for all tangent vectors \(X, Y \in T_yM\) for almost all $y \in V_x$.
\end{defn}

\begin{rem}
This condition is equivalent to:  there is an open cover \(\lbrace U_{\alpha} \rbrace\) of \(M\), smooth metrics \(g_{\alpha} \in \met(U_{\alpha})\) and real constants \(C_{\alpha} > 0\) such that
\[
\frac{1}{C_{\alpha}} g_{\alpha} (X, Y) \leq g(X, Y) \leq C_{\alpha} g_{\alpha} (X, Y)
\]
for all tangent vectors \(X, Y \in T_yM\) for almost every $y \in U_\alpha$.
\end{rem}

\begin{rem}
By employing a partition of unity argument, we can patch together the metrics \(g_{\alpha}\) to produce a globally defined metric \(h\). If the constants \(C_{\alpha}\) are uniformly bounded above and away from zero, then our rough metric will be globally comparable to \(h\). This is automatic whenever \(M\) is compact. If \(M\) is not compact, then this need not be true. Previously a Harnack inequality and heat kernel estimates have been obtained when the metric is globally comparable to a complete metric with bounded below curvature \cite{SC}. Here we do not require the assumptions of global comparability.
\end{rem}

Given a rough metric, we may locally define a \(g_{\alpha}\)-self-adjoint, measurable section \(A_{\alpha} : U_{\alpha} \to T^{1,1}U_{\alpha}\) by
\[
g(X, Y) = g_{\alpha} (A_{\alpha} (X), Y).
\]
Moreover, via the expression $d\omega_g(x) = \sqrt{\det g_{ij}(x)}\ dx$, where $dx$ is the pullback
Lebesgue measure inside precompact $U_\alpha$, we obtain a well-defined volume measure. 
A set $S \subset M$ is measurable if and only if it is $\omega_g$-measurable. This yields 
an $\Lp{p}$-theory over the $T^{p,q}M$ tensor bundle simply via the expression:
$$||\xi||_p^p := \int_{M} |\xi|_{g}^p\ d\omega_g < \infty$$
for $p \in [1, \infty)$ and via
$$||\xi||_\infty := \inf\set{C: |\xi|_g \leq C\ \text{a.e.}}.$$
All spaces we consider are complex-valued function spaces, 
which are obtained from the real setting by complexification.

\section{Laplacians for rough metrics}


For a smooth function \(f \in \Ck{\infty}(M)\), write \(\nabla f = df\) for the differential of \(f\). 
The differential \(\nabla : \Ck{\infty}(M) \to \Ck{\infty}(T^{\ast} M)\), maps smooth functions 
\(\Ck{\infty}(M)\) to smooth one-forms \(\Ck{\infty}(T^{\ast} M)\). 
We emphasise that this object is only dependent on choice of differentiable structure on $M$,
which we assume to be fixed and a part of the data specifying $M$ as a manifold. 
It is not dependent on the metric.

Recall that in the classical setting of a smooth $g$,  the Laplacian acting on functions 
is given  via the expression $\Delta_g = - \tr \nabla^2$, where $\nabla^2 = \nabla^{T^\ast M} \comp \nabla$
with $\nabla^{T^\ast M}$ the Levi-Civita connection of $g$. Indeed, for rough metrics, 
or even for a continuous metric, we cannot talk of the Levi-Civita connection. 
Therefore, we understand this operator in an appropriate weak sense as follows.

First, we observe that for a rough metric $g$, the operator
$\nabla_2 := d_2: \Ck{\infty} \cap \Lp{p}(M) \to \Ck{\infty} \cap \Lp{p}(T^\ast M)$
is closable, as well as $\nabla_c := d_c = d_2$ with $\dom(d_c) = \Ck[c]{\infty}(M)$,
for $p \in [1, \infty)$. A proof of this statement is given in Proposition 3.10 in \cite{BRough}, 
which reduces to covering the manifold via precompact locally comparable
charts and noting that $d$ commutes with pullbacks inside each such chart. 
Consequently, we can define first-order Sobolev spaces
$$ \SobH[0]{1}(M) := \dom(\bar{\nabla_c})\quad \text{and}\quad  \SobH{1}(M) := \dom(\bar{\nabla_2}),$$
with the Sobolev norm $||u||_{\SobH{1}} = ||u||_2 + ||\nabla u||_2$
as function spaces $\SobH[0]{1}(M) \subset \SobH{1}(M) \subset \Lp{2}(M)$.
We remark that without closability, if these spaces were obtained
via a completion with respect to the Sobolev norm, we could not 
assert that they were spaces of functions.

Another consequence of the closability, coupled with the 
fact that $\Ck[c]{\infty}(M)$ are dense in $\Lp{2}(M)$ 
is that $\nabla_2$ and $\nabla_c$ are densely-defined
operators. Therefore, operator theory yields 
that $\adj{\nabla_2}$ and $\adj{\nabla_c}$ exist
as densely-defined, closed operators.


In the case of a smooth $g$ that is also complete, we
always have that $\SobH[0]{1}(M) = \SobH{1}(M)$. 
Then, since $\nabla_c = \nabla_2$, we obtain 
a unique Laplacian $\Delta_g = \adj{\nabla_c} \close{\nabla_c} = \adj{\nabla_2}\close{\nabla_2}$.
We will emphasise at this moment that this does not mean that 
the Laplacian is essentially self-adjoint on $\Ck[c]{\infty}(M)$; in 
fact, the only general statement that can be made is that
$\SobH{2}(M) \subset \dom(\Delta_g)$. The 
case of essential self-adjointness can
be obtained under a uniform lower bound on Ricci
curvature (see \cite{BDensity}) but it
is not known to the authors whether this condition is sharp.

It is also useful to obtain $\Delta_g$ via its ``energy''. 
That is,  we consider the expression 
$$\sE_g(u,v) = \inprod{\nabla u, \nabla v} = \int_M\ g(\nabla u, \nabla v)\ d\omega_g,$$
where $\dom(\sE_g) = \SobH[0]{1}(M) = \SobH{1}(M)$.
The operator $\Delta_g$ is now obtained via the Riesz representation theorem
or Lax-Milgram theorem.
Note that in our setup, $\nabla_c$ is independent of the geometry,
but it is in taking its adjoint $\adj{\nabla_c}$ where the
metric information $g$ becomes of consequence. If we were
to define the gradient $\grad u =  (\nabla u)^\sharp = g(\nabla u, \cdot)$
then this object would see the metric. However,
we note that 
$\sE_g(u,v) = \inprod{\grad u, \grad v}$
and moreover, $\Delta_g = \adj{\grad_c} \close{\grad_c}$, 
so it does not matter which definition we pick.
We prefer the former picture for the
simple fact that in the latter picture, the metric information
and topological information are intertwined whereas 
in our case we have two operators in which one which clearly only depends
on the differential structure and the other which sees the geometry $g$.

When $g$ fails to be complete, it may be that  
$\SobH[0]{1}(M) \subsetneqq  \SobH{1}(M)$
if $g$ is not assumed to be complete. 
In that case, we obtain a \emph{Dirichlet Laplacian} and  a \emph{Neumann 
Laplacian} corresponding to which space we pick 
to consider the associated energy.
We retain this language from the world of boundary 
value problems because there, when $M = \interior\Omega$
for a bounded domain $\Omega$ (say with Lipschitz boundary), we have that
$\SobH[0]{1}(M)$ defines the energy for
the Laplacian considered in the Dirichlet problem
and $\SobH{1}(M)$ defines the energy 
for the Laplacian considered for the Neumann problem.

In our approach, which is valid even in the setting
for smooth $g$, we proceed slightly more generally. 
Inspired by the classical setting, we define
the following.
 
\begin{defn}[$(g,\cW)$-Laplacian]
Let \(g\) be a rough metric on a smooth manifold \(M\)
and $\cW \subset \SobH{1}(M)$ be a closed subspace
of $\SobH{1}(M)$ such that $\Ck[c]{\infty}(M) \subset \cW$
and $\Ck{\infty} \cup \cW$ dense in $\cW$ 
in the norm $\norm{\cdot}_{\cW} = \norm{\cdot}_{\SobH{1}}$. 
The \emph{$(g,\cW)$-Laplacian}
\(\Delta_{g,\cW} : \Lp{2}(M) \to \Lp{2}(M)\) is defined by
$\Delta_{g,\cW} u = \adj{\nabla_{\cW}} \close{ \nabla_{\cW}} u,$
where  $\nabla_{\cW} = \nabla$ with $\dom(\nabla_{\cW}) = \cW$. 
The domain of the operator $\Delta_{g,\cW}$ is then given by
$$\dom(\Delta_{g,\cW}) = \set{u \in \SobH[0]{1}(M): \modulus{\inprod{\nabla u, \nabla v}} \lesssim C_u \norm{v}\ \forall v \in \SobH[0]{1}(M)}.$$
\end{defn}

By construction, the  operators $\Delta_{g,\cW}$ 
are densely-defined, closed, self-adjoint and satisfy 
$\dom(\sqrt{\Delta_{g,\cW}}) = \cW$.
It is difficult to see how to obtain this from the expression for $\Delta_{g,D}$
Rather, this operator is constructed via the Riesz representation
theorem (or the theorem of Lax-Milgram) by considering the
energy $\sE_{g,\cW}(u,v) = \inprod{\nabla u, \nabla v}$ 
with $\dom(\sE_{g,\cW}) = \cW$ written exactly as we did for the smooth case.
These are routine facts from operator theory in far more general 
settings than what we consider here. See for instance 
\cite{Yosida, Kato}.

Let us remark on why we work at this level 
of generality. First, observe that the case of $\cW = \SobH[0]{1}(M)$
yields the Dirichlet Laplacian
and $\cW = \SobH{1}(M)$ yields the Neumann counterpart.
However, beyond these two obvious choices, 
there are many interesting spaces $\cW$ that
can be considered. These are best
seen emerging from boundary value problems.
As a guiding example, let $\Omega \subset \R^n$
be a smooth (or Lipschitz) bounded domain, and let $\Sigma \subset \partial \Omega$
be an open subset of the boundary $\partial \Omega$.
On letting $\Tr: \SobH{1}(\Omega) \to \SobH{\frac{1}{2}}(\partial \Omega)$
be the trace map to the boundary, define 
$\cW_{\Sigma} = \set{u \in \SobH{1}(\Omega): \spt (\Tr u) \subset \close{\Sigma}}$. 
It is clear that $\Ck[c]{\infty}(\Omega) \subset \cW_{\Sigma}$ and
it can be shown that $\cW_{\Sigma} \subset \SobH{1}(\Omega)$
is a closed subset. 
The interpretation here is that the part $\Sigma$ 
specify Neumann boundary conditions whereas $\Omega \setminus \Sigma$
specify Dirichlet boundary conditions. Therefore
these are mixed boundary value problems as considered
by  Axelsson (Rosén), Keith and McIntosh in \cite{AKM2}.

Now, we proceed to define what we mean by a solution to the heat
equation, which becomes the central theme of what is to follow.

\begin{defn}[Solution to the heat equation]
A function \(u \in \Ck{1}((0,\infty),  \dom(\Delta_{g,\cW})\) solves the \emph{heat equation} for 
the heat equations
with initial condition $u_0 \in \Lp{2}(M)$  if we have that
\begin{enumerate}[(i)]
\item  $\partial_t u (\cdot, t) = \Delta_{g,\cW} u(\cdot, t)$ for $t \in (0, \infty)$ and 
\item $\lim_{t \to 0} u(\cdot, t) = u_0$ in $\Lp{2}(M)$.
\end{enumerate}
\end{defn}

For any solution \(u\), of the heat equation, we have the representation formula
\[
u(x,t) = (e^{-t \Delta_{g,\cW}}u_0)(x)
\]
for almost every $x \in M$.
 
\begin{rem}
In the compact case with a smooth metric, we can consider $\Delta = -\tr \nabla^2$,
and there,  a solution is often formulated to mean $u \in \Ck{\infty}( M \times (0, \infty))$
with $\partial_t u(x,t) = \Delta u(x,t)$ with 
$\lim_{t \to 0} u(x,t) = u_0 \in \Ck[c]{\infty}(M)$.
In this case, $\Delta_{g,\cW} = \Delta_{g,\theta}$ is independent
of $s$ and moreover, since $\Ck{\infty}(M) \subset \Lp{2}(M)$
this notion of solution is stronger than our notion of solution.

\note{Fix this, find out what is meant by a definition in the classical setting 
for noncompact manifolds when they construct this heat kernel via a limit,
and show that either our notion of a function is sufficiently weak to capture the classic shit
or it is necessary for us to formulate the problem in this way because
we don't have the classical methods at hand}
\end{rem}

\section{Existence and positivity of the heat kernel}

\begin{defn}[Heat kernel]
Let $u \in \Ck{1}(\R_+, \cW)$ be a solution to the 
Dirichlet heat equation
for $\Delta_{g,\cW}$ with initial data $u_0 \in \Lp{2}(M)$. Then, a 
measurable symmetric map $\hk^{g,\cW}_{\mdot}: \R_+ \times M \times M \to \R$
is a heat kernel if 
$$u(t,x) = \int_{M} \hk^{g,\cW}_t(x,y) u_0(y)\ d\omega_g(y),$$
with $\lim_{t \to 0} \hk_t^{g,\cW}(\mdot,y) \to \delta(y)$, 
where $\delta(y)$ is the Dirac-delta function.
\end{defn}

In the case of a smooth metric $g$, a typical construction for
the heat kernel is to construct the so-called \emph{minimal} heat kernel.
This is done by taking smooth domains $\Omega_j$, 
each of which are precompact and satisfying $\close{\Omega_j} \subset \Omega_{j+1}$. 
Inside each domain, one can solve the Dirichlet problem
to obtain heat kernels $\hk^{g,j}_{t}$, each of which
satisfies 
$$ \int_{M} \hk^{g,j}(x,y)\ d\omega_g(y) < 1.$$
Then, one can make sense of the limit $\lim_{j \to \infty} \hk^{g,j}(x,y)$
in the compact-open topology to obtain a heat kernel $\hk^{\min}_t$.
See Chapter VIII in \cite{Chavel} for the details
of this construction.

We refrain from considering this approach in the rough setting 
for the reason that this object may fail to be unique (uniqueness
is known for smooth complete $g$ with 
uniform lower bounds on Ricci curvature) and more seriously, the following
Varadhan's asymptotics may fail: 
$d^2(x,y) = \lim_{t \to 0} 4t \log \hk^{\min}_t(x,y)$,
which is a motivating factor for this paper as mentioned in the 
introduction.

Consequently, we can instead consider heat kernels
associated to the operator $\Delta_{g,\cW}$ in $\Lp{2}$
via the Riesz representation theorem as described in Theorem 
5.2.1 in \cite{Davies}. As aforementioned, in the smooth case,
at least for the heat kernel corresponding to the 
Dirichlet Laplacian, we obtain the desired small time
asymptotics. 

There is a large class of rough metrics
for which we know that the heat kernel exists
and for which Varadhan's formula holds. 
Take $h$ smooth and let $\psi:M \to M$
be a Lipeomorphism (that is, a locally bi-Lipschitz map).
Define $g = \pullb{\psi}h$, then, $g$ 
also induces a length structure with 
distance given by $d_g(x,y) = d_h(\psi(x), \psi(y))$.
Moreover, a calculation gives
that $\hk^{g}_t(x,y) = \hk^{h}_t(\psi(x), \psi(y))$
(or more generally for $\hk^{g,\cW}_t(x,y)$). 
On combining the fact that $\hk^h_t$ satisfies
Varadhan's formula, it is easy to see that
so does $\hk^{g}_t$. In fact, we 
note that $\hk^g`_t \in \Ck{0,1}(M \times M)$,
which is of higher regularity than we we
obtain for general rough metrics.

In what follows, we adapt the key idea in 
the proof of Theorem 5.2.1 in \cite{Davies}
to our setting. Due to the more general 
nature of our problem, we are forced to 
establish a number of a priori facts which 
are more or less immediate in the smooth case.
What we present here was initially adapted for the
compact case in \cite{BRough}, but further analysis
has shown that we can make it work in the general.

With this terminology, we present the fundamental estimate we
require is the following \emph{weak Harnack-type inequality}.
\begin{itemize}
\item[] At each $x \in M$ and $t > 0$, there exists a precompact
open set $U_x$, a $\delta_t \geq 0$ and a constant $C = C(t, U_x)  > 0$ such that
\begin{equation}
\tag{H}
\label{eq:Harnack}
\esssup_{y \in U_x} u(y,t) 
	\leq \essinf_{y \in U_x} u(y, t + \delta_t)
\end{equation}
\end{itemize}

In the compact case, even for a
rough metric, such an estimate can be
obtained with $C(t,U_x)$ precisely quantified
in terms of the curvature of a nearby smooth metric.
However, the key observation to pass from the compact analysis
to the general setting we present here was to note that
the estimates we require are purely local. In 
what is to follow, we will see that the constant can be
crude, it simply allows us to assert the existence and regularity
of $\hk^{g,\cW}_t$, but the finer properties can still 
be extracted by operator theory, in particular, from the 
fact that $t \mapsto e^{-t\Delta_{g,\cW}}$ is a semigroup.

\begin{prop}
\label{prop:MainRed}
Suppose that \eqref{eq:Harnack} holds. Then, the heat
kernel $\hk^{g,\cW}_{\mdot}: \R_{+} \times M \times M \to \R$
exists. Moreover, $\hk^{g,\cW}_t(x,y) > 0$ for
every $t > 0$ and almost every $x,\ y \in M$. 
\end{prop} 
\begin{proof}
We outline the steps of the construction of the heat kernel
noting that the pointwise expressions from here on should be
understood in an almost-everywhere sense.
\begin{enumerate}[(i)]
\item For $u \in \Ck{\infty}$, it is readily verified that 
	$\modulus{ \nabla \modulus{u}} \leq \modulus{ \nabla u}.$
	In particular, this means that 
	for $u \in \Ck{\infty} \intersect \cW$
	$\norm{\nabla \modulus{u}} = \norm{\nabla u},$
	and therefore, whenever $u \in \cW$
	we have that $\modulus{u} \in \cW$ 
	with this estimate holding.

\item 	By construction $\dom(\sqrt{\Delta_{g, \cW}}) = \cW$
	and $\norm{\sqrt{\Delta_{g,\cW}} f} = \norm{\nabla f}$
	for all $f \in \cW$ and therefore, 
	$\norm{\sqrt{\Delta_{g, \cW}} \modulus{u}} 
		\leq \norm{\sqrt{\Delta_{g, \cW}} u}$. 
	By the Buerling-Deny condition (c.f.  Corollary 2.18(2) in  \cite{El-Maati}), 
	this yields that for $\Lp{2}(M) \ni f \geq 0$
	we have that  $e^{-t \Delta_{g,\cW}} f \geq 0$. That is, the semigroup 
	$e^{-t \Delta_{g,\cW}}$ is positive preserving.  

\item Now, left $f \in \Lp{2}(M)$, and write
	$f = f_+ - f_-$, where $f_{\pm} = \max\set{0,\pm f}$.
	It is clear that $f_{\pm} \in \Lp{2}(M)$
	and that $\modulus{f} = f_+ + f_-$.
	By the fact that we have shown $e^{-t\Delta_{g,\cW}}$ 
	is positive preserving, this means that
	for $u_{\pm} (x,t) = e^{-t\Delta_{g,\cW}} f_{\pm} \geq 0$	
	and we have that $u(x,t) = e^{-t \Delta_{g,\cW}} f = u_+(x,t) - u_-(x,t)$.

\item  Since $u_{\pm}(x,t) \geq 0$ are positive solutions,
	using inequality \eqref{eq:Harnack}, we have a precompact $U_x$ 
	with 
	\begin{align*} 
	\modulus{e^{-t \Delta_{g,\cW}}f(x)}  &= \modulus{u(x,t)} = u_+(x,t) + u_-(x,t) \\
		&\leq C(t,U_x) (u_+(y,t+\delta_t) + u_-(y,t+\delta_t)) \\
		&= C(t,U_x) \modulus{u(y,t + \delta_t)}
		= C(t,U_x)  \modulus{e^{-(t + \delta_t) \Delta_{g,\cW}}f(y)}
	\end{align*} 
	for almost-every $y \in U_x$.
	On integrating both sides over $U_x$ (which has $\omega_g(U_x) < \infty$
	by precompactness and the properties of the measure $\omega_g$) with respect
	to the variable $y$,
	we have that
	\begin{align*} 
	\modulus{e^{-t \Delta_{g,\cW}}f(x)} 
		&\leq \frac{C(t,U_x))}{\omega_g(U_x)} \int_{U_x} \modulus{e^{-(t + \delta_t) \Delta_{g,\cW}}f(y)}\ d\omega_g(y) \\
		&\leq \frac{C(t,U_x)}{\omega_g(U_x)^{\frac{1}{2}}} \norm{e^{-(t + \delta_t) \Delta_{g,\cW}f}},
	\end{align*}
	where the second inequality follows from the Cauchy-Schwartz
	inequality and since $\norm{\cdot}_{\Lp{2}(U_x)} \leq \norm{\cdot}_{\Lp{2}(M)}$.

\item By the self-adjointness of $\Delta_{g,\cW}$, via functional calculus 
	we obtain that $\norm{e^{-s\Delta_{g,\cW}}f} \leq \norm{f}$
	uniformly in $s$  and therefore, we obtain that
	$$\modulus{e^{-t \Delta_{g,\cW}}f(x)} \leq 
		\frac{C(t,U_x)}{\omega_g(U_x)^{\frac{1}{2}}} \norm{f}.$$
	This exactly says that $f \mapsto (e^{-t\Delta_{g,\cW}}f)(x) \in \adj{\Lp{2}(M)}$
	for every $f \in \Lp{2}(M)$ and therefore, by the Riesz Representation theorem,
	we obtain an $a_{t,x} \in \Lp{2}(M)$ such that
	$$ e^{-t \Delta_{g,\cW}}f(x) = \inprod{a_{t,x},f}.$$

\item Again, by self-adjointness, we have that $e^{-t \Delta_{g,\cW}}$
	is bounded self-adjoint, and
	on combining this with the semigroup property 
	$e^{-t\Delta_{g,\cW}} = e^{-\frac{t}{2} \Delta_{g,\cW}}e^{-\frac{t}{2} \Delta_{g,\cW}}$, 
	it is straightforward to check that 
		$$\hk^{g,\cW}_t(x,y) = \inprod{a_{\frac{t}{2}, x}, a_{\frac{t}{2}, y}},$$
	defines a measurable symmetric function for which 
	$$e^{-t \Delta_{g,\cW}}f = \int_{M} \hk^{g,\cW}_t(x,y)f(y)\ d\omega_g(y).$$
	Since $e^{-t \Delta_{g,\cW}}$
	is positive preserving which shows that $\hk_t^{g,\cW}$
	is real-valued. Moreover, since $e^{-t\Delta_{g,\cW}}f \to f$
	in $\Lp{2}$, we have that 
	$\lim_{t \to 0} \hk^{g,\cW}_t(\mdot,y) \to  \delta(y)$
	in the sense of distributions. This
	shows that $\hk^{g,\cW}_t$ is the heat kernel.

\item Lastly, we show that $\hk^{g,\cW}_t(x,y) > 0$  almost-everywhere.
	We reason by contradiction so assume that there
	is a time $t > 0$ and a measurable set $A \subset M$ with $\omega_g(Z) > 0$
	such that $\hk^{g,\cW}_t(x,y) = 0$ for $x \in A$.
	Since $M = \union_{n} \Omega_n$ with each 
	$\Omega_n$ precompact, we can find an integer
	$k$ for which $B = A \intersect \union_{n=1}^k \Omega_k$ 
	satisfies $0 < \omega_g(B) < \infty$. Then,
	we have that $\ind{B} \in \Lp{2}(M)$ and 
	on using the self-adjointness of $\Delta_{g,\cW}$,
	we note
	\begin{multline*}  
	\norm{e^{-\frac{t}{2} \Delta_{g,\cW}}\ind{B}}^2 
		 = \inprod{e^{-t \Delta_{g,\cW}} \ind{B},\ind{B}} \\
		= \int_{B} \hk^{g,\cW}_t(x,y)\ind{B}(y)\ d\omega_g(y) + 
			\int_{M \setminus B} \hk^{g,\cW}_t(x,y)\ind{B}(y)\ d\omega_g(y).
	\end{multline*}
	Now, note that the first integral vanishes
	since we assume $\hk^{g,\cW}_t$ is $0$ on $B$
	and the second since $\ind{B} = 0$ on $M \setminus B$.
	Therefore, we find that $\norm{e^{-\frac{t}{2} \Delta_{g,\cW} \ind{B}}} = 0$.
	This, along with the fact that we've shown $\hk^{g,\cW}_s(x,y) \geq 0$ 
	for each $s > 0$ yields that $\hk^{g,\cW}_{\frac{t}{2}}(x,y) = 0$ for 
	almost every $x,\ y \in B$, and on 
	iterating this procedure, we obtain that
	$\norm{e^{-\frac{t}{2^m} \Delta_{g,\cW} \ind{B}}} = 0$.
	Thus, we have that $\lim_{s\to 0} e^{-s \Delta_{g,\cW}} \ind{B} = 0$, 
	but by the non-negative self-adjointness of
	$\Delta_{g,\cW}$, we know that $\lim_{s \to 0} e^{-t \Delta_{g,\cW}} u = u$
	for any $u \in \Lp{2}(M)$.  This is a contradiction
	which shows that $\hk^{g,\cW}_t(x,y) > 0$ almost everywhere.
\end{enumerate} 
\end{proof}

\begin{rem}
We note that the Harnack-type estimate that we assume is very 
weak, i.e., it is not defined on cylinders or even 
parabolic cylinders. The proof shows that the existence 
of the heat kernel only requires such an estimate. 
As we shall see in the next section, a stronger
estimate is required for regularity of solutions. 
\end{rem}

%Lastly, in this section, we address the question of 
%the positivity of the heat kernel. For that, 
%we use the language of the convergence 
%increasing self-adjoint non-negative operators.
%We take inspiration for this proof
%from the proof of Theorem 2.1.6 in \cite{Davies}
%by Davies. 
%
%\begin{prop}
%\label{prop:hkpos}
%The heat kernel $\hk^{g,\cW}_t(x,y) > 0$ for almost every 
%$x,\ y \in M$.
%\end{prop}
%\begin{proof}
%Let $\Omega$ be a precompact open set in $M$ with $\partial \Omega$
%smooth. Let $\Delta_{g,\cW,k} = \Delta_{g,\cW} + k \ind{M \setminus \Omega}$
%and let $\sE_{g,\cW,k}$ be it associated sesquilinear
%form, given by 
%$$\dom(\sE_{g,\cW,k}) = \dom(\sqrt{\Delta_{g,\cW,k}}) 
%	= \dom(\sqrt{\Delta_{g,\cW}}) = \cW,$$
%and $\sE_{g,\cW,k}(u,v) = \inprod{\sqrt{\Delta_{g,\cW,k}} u, \sqrt{\Delta_{g,\cW,k}}v}$. 
%
%Recall that the limiting form exists by Theorem 1.2.2 
%in \cite{Davies} and is given by: 
%$u \in \dom(\sE_{g,\cW,\infty})$ if and only if
%$\sE_{g,\cW,\infty}(u,u) = \lim_{k\to\infty} \sE_{g,\cW,k}(u,u) < \infty$. 
%To compute this domain, take $u \in \Ck{\infty}(M) \intersect W$
%and 
%\begin{multline*} 
%\sE_{g,\cW,k}(u,u) = \norm{\sqrt{\Delta_{g,\cW,k}}u}^2
%	= \inprod{\Delta_{g,\cW,k}u,u} \\
%	= \norm{\nabla u^2} + k \inprod{\ind{M \setminus \Omega} u, u}
%	= \norm{\nabla u^2} + k \norm{\ind{M \setminus \Omega} u}^2.
%\end{multline*}
%Now, by the characterisation of $\dom(\sE_{g,\cW,\infty})$
%we see that we must have $u = 0$ on $M \setminus \Omega$,
%and since $\cW \subset \SobH{1}(M)$, we have that
%$u\rest{\Omega} \in \SobH[0]{1}(\Omega)$. 
%Since we've established that $e^{-t{\Delta_{g,\cW}}}$
%is positive preserving, we similarly have that 
%$e^{-t {\Delta_{g,\cW,k}}}$ is positive preserving
%by repeating the argument in Proposition \ref{prop:MainRed}
%noting that 
%$\norm{\sqrt{\Delta_{g,\cW,k}}u}^2 
%	= \norm{\nabla u^2} + k \norm{\ind{M \setminus \Omega} u}^2$
%and hence, for $f \geq 0$ with $f \in \Lp{2}(M)$,
%$ e^{-t{\Delta_{g,\cW,k}}}f \leq e^{-t{\Delta_{g,\cW}}}f.$
%Taking limits, we obtain that 
%$$e^{-t{\Delta_{g,\SobH[0]{1}(\Omega)}}}f
%	\leq e^{-t{\Delta_{g,\cW}}}f,$$
%where the left hand expression is extended
%to the whole of $M$ by $0$ 
%and $\Delta_{g,\SobH[0]{1}(\Omega)}$ is the Dirichlet
%Laplacian inside $\Omega$.
%\note{We need to ensure that this limit exists 
%for the exponential, find a citation or justify 
%why we can do this. Davies does a similar thing in Theorem 2.1.6}
%
%Now, since we assumed that $\Omega$ is precompact,
%there exists a smooth metric $h$ such that
%$g(u,v) = h(Au,v)$ with $A \in \Lp{\infty}(T^{1,1}M, h)$
%and $d\omega_g = \sqrt{\det A} d\omega_h$.
%Therefore, we have that
%$$ \norm{\nabla  u}_{\SobH[0]{1}(\Omega,h)} 
%	\lesssim \norm{\nabla u}_{\SobH[0]{1}(\Omega,g)}
%	\lesssim \norm{\nabla u}_{\SobH[0]{1}(\Omega,h)},$$
%and so we have 
%\begin{multline*}
%\set{0} = \nul(\Delta_{h, \SobH[0]{1}(\Omega,h)}) 
%	= \nul\cbrac{\sqrt{\Delta_{h, \SobH[0]{1}(\Omega,h)}}} \\
%	= \nul\cbrac{\sqrt{\Delta_{g, \SobH[0]{1}(\Omega,g)}}}
%	= \nul(\Delta_{g, \SobH[0]{1}(\Omega,g)}).
%\end{multline*}
%Moreover, we have that 
%the first eigenvalues of the Laplacians satisfy
%$$0 < \lambda_1(\Delta_{h,\SobH[0]{1}(\Omega,h)}) \leq 
%	C' \lambda_1(\Delta_{g, \SobH[0]{1}(\Omega,g)})$$
%where the constant $C' > 0$ depends on $A$
%and hence, we have that $e^{-t\Delta_{g,\SobH[0]{1}(\Omega,h)}}\ind{\Omega} > 0$.
%Since we know that $e^{-t\Delta_{g,\SobH[0]{1}(\Omega,h)}}$
%is positive preserving, this proves
%the claim.
%\note{Check through this argument, make sure it holds water.}
%\end{proof}  

\section{The Harnack inequality and regularity}
\label{sec:harnack}


In this section, we demonstrate that a
 stronger form of the  Harnack 
estimate than \eqref{eq:Harnack} holds for the rough 
metrics we consider. The estimate \eqref{eq:Harnack} 
will then be a consequence of this estimate.
To describe this estimate, fix a collection of 
locally comparable charts $\psi_x:V_x \to \R^n$ denote
for each $x \in M$ such that $\psi_x(V_x) = B_{r_x}$,
where $B_{r_x}$ is a ball of radius $r_x > 0$. 
Then, let
$\delta_x(\cdot, \cdot) = \pullb{\psi_x}\delta(\cdot, \cdot)_{\R^n}$
denote the pullback metric with $\Leb_x = \pullb{\psi_x}\Leb$
and $d_x(y,y') = \modulus{\psi_x(y') - \psi_x(y)}_{\R^n}$.
Recall that there is a $C_x \geq $
with $C_x^{-1}  \modulus{u}_{\delta_x} \leq \modulus{u}_g \leq C_x \modulus{u}_{\delta_x}$
almost-everywhere in $V_x$.

Fix $0 < \kappa < \tau < \infty$ and an $0 < \epsilon < \kappa$,
and for $x \in M$ and $t \in (\kappa, \tau)$, define
\begin{equation}
\label{eq:paradef}
\begin{split} 
Q^-_{(x,t)}(\kappa, \tau, \epsilon) &= \set{(y,s) \in V_x \times (\kappa,\tau): s \in \cbrac{t - \frac{3}{4}\epsilon^2, t - \frac{1}{4} \epsilon^2},\ 
	d_x(x,y) < \frac{1}{2} \epsilon} \\
Q^+_{(x,t)}(\kappa, \tau, \epsilon) &= \set{(y,s) \in V_x \times (\kappa,\tau): s \in \cbrac{t + \frac{3}{4}\epsilon^2, t + \epsilon^2},\ 
	d_x(x,y) < \frac{1}{2} \epsilon}
\end{split}
\end{equation}
Then, we prove the following. 

\begin{thm}
\label{thm:Harnack}
Let the dimension of $M$ be $3$ or greater 
and let $u(x,t) \geq 0$ be a 
solution to the $\Delta_{g,\cW}$ heat equation 
that is non-negative in $(\kappa, \tau)$. 
Then, there exists $\gamma = \gamma(n, C_x, \kappa, \tau) > 0$ such that 
$$ \sup_{(y,s) \in Q^-_{(x,t)}(\kappa, \tau)(\epsilon)} u(y, s) 
	\leq \gamma \inf_{(y,s) \in Q^+_{(x,t)}(\kappa,\tau,\epsilon)} u(y,s)$$
for all $\epsilon < \min\set{ \sqrt{t - \kappa}, \sqrt{\tau - t}, r_x}$.
\end{thm} 

In the compact setting, such an inequality 
is immediate from the work of \cite{SC}. There, 
he proves such estimates for general operators
$L = -a \divv A \nabla$, where $a \in \Lp{\infty}(M)$
and $A \in \Lp{\infty}(T^{1,1}M)$ and symmetric
for smooth metric $h$ with a uniform lower bound
on Ricci curvature.
The key to note is that a rough metric in the 
compact setting is globally comparable to a smooth one, 
i.e., there exists a global constant $C \geq 1$ 
and a 
such that
$$ C^{-1} \modulus{u}_{h} \leq \modulus{u}_g \leq C \modulus{u}_{h}$$
for $u \in T_x M$ for almost-every $x \in M$.
By the virtue of compactness, it is immediate
that the Ricci curvature of $h$ is bounded below
by a uniform constant. 
In fact, this procedure also works 
for rough metrics $g$ which 
are uniformly close to some metric $h$ with 
lower bound on Ricci curvature in the sense
we have just written.

We emphasise that the goal in \cite{SC}
is to quantify the constants $C(t,U_x)$
appearing in Theorem \ref{thm:Harnack}. In what we present here, we do not have
the ability to control these constants. However, as
Proposition \ref{prop:MainRed} illustrates, this is not necessary 
in order to obtain the existence of the heat kernel.
Quantifying such estimates is, however, still 
extremely important and useful when it can be done.
For instance, Proposition \ref{prop:MainRed} 
only asserts that $\hk^{g}_t \in \Ck{0}(M \times M)$
but if $g$ was uniformly close to a $h$
with a lower bound on its Ricci curvature, then
there would exist an $\alpha$ depending on this bound
with $\hk^{g}_t \in \Ck{\alpha}(M \times M)$.
\note{Verify this is true}

The main way in which we will proceed is to use the
results in \cite{CS}. There, they analyse
degenerate equations via the methods of harmonic analysis.

Throughout the remainder of this section,
we assume that $u(\cdot, \cdot): M \times \R_+$ 
is a solution to the $\Delta^{g,\cW}$ heat equation.
Moreover, let $g(u,v) = \delta_x(A_x u,v)$
and $a_x = \sqrt{\det A_x}$ so that
$\omega_x(y) = a_x(y)d\Leb_x(y)$, and note
that:
\begin{equation}
\label{eq:constbd}
\begin{split}
&C_x^{-2} \modulus{u}_{\delta_x} \leq \delta_x(A_x u, u) \leq C_x^2 \modulus{u}_{\delta_x} \\
&C_x^{-\frac{n}{2}} \leq a_x \leq C_x^{\frac{n}{2}}.
\end{split}
\end{equation}
See Section 3.3 in \cite{BRough} for details. 

\begin{lemma}
\label{eq:soln1}
The solution $u$ satisfies the following: for all $v \in \SobH[0]{1}(V_x,g)$, 
	\begin{equation} 
	\label{eq:testfn1}	 
	\inprod{a_x \partial_t u, v}_{\Lp{2}(V_x, \delta_x)} 
		= \inprod{ B_x \nabla u, \nabla v}_{\Lp{2}(V_x, \delta_x)},
	\end{equation}
where $B_x = a_x A_x$. Moreover,
\begin{equation}
\label{eq:coeff}
C_x^{-2} a_x \modulus{\xi}_{\delta_x}^2 \leq \delta_x(B_x \xi, \xi) 
	\leq C_x^2 a_x \modulus{\xi}_{\delta_x}^2
\end{equation} 
for every $\xi \in T_xM$ and almost everywhere in $V_x$. 
\end{lemma}
\begin{proof}
Since the solution $u \in \dom(\Delta_{g,\cW}) \subset \Lp{2}(M)$, we
note that in particular, for any $v \in \Ck[c]{\infty}(M)$, 
$\inprod{\partial_t u, v} = \inprod{\nabla u, \nabla v}$.
Choosing $\spt v \subset V_x$, we note that $\inprod{\cdot, v} = \inprod{\cdot, v}_{\Lp{2}(V_x,g)}$
and hence,
$$\inprod{\partial_t u, v} = \int_{V_x} \partial_t u\conj{v}\ d\omega_g 
	= \int_{V_x} a_x \partial_t  u \conj{v}\ d\Leb_x
	= \inprod{\partial_t (a_x u), v}_{\Lp{2}(U_x, \delta_x)},$$
where the last equality follows from the fact that 
$a_x(y) \partial_t u(y,t) = \partial_t (a_x(y)u(y,t).$
Similarly,
$$\inprod{\nabla u, \nabla v} 
	= \int_{V_x} g(\nabla u, \nabla v)\ d\omega_g
	= \int_{V_x} \delta_x(a_x A_x \nabla u, \nabla v)\ d\Leb_x
	= \inprod{B_x \nabla u, \nabla v}_{\Lp{2}(U_x, \delta_x)}.$$
The estimate on $B_x$ follows immediately from
\eqref{eq:constbd}.
\end{proof} 

For $0 < \kappa < \tau < \infty$ fixed, let
define $u_\kappa(x,t) = u(x,t+\kappa)$. Then we note the following. 
\begin{lemma}
\label{lem:shiftsol}
We have that: 
\begin{enumerate}[(i)] 
\item $u_\kappa$ is a solution to the $\Delta_{g,\cW}$ heat equation, 
\item there is a constant $C_\kappa \geq 0$
	which depends on $\kappa$ 
	such that $\norm{\nabla u_\kappa} \leq C_\kappa$.
\end{enumerate} 
\end{lemma}
\begin{proof}
Note that by definition, and using the semigroup property, 
$$u_\kappa(x,t) = e^{-(t + \kappa)\Delta_{g,\cW}} u_0 
	=e^{-t \Delta_{g,\cW}}e^{-\kappa\Delta_{g,\cW}}u_0
	=e^{-t \Delta_{g,\cW}}u(x,\kappa).$$ 
Therefore, it is immediate that $u_\kappa$ is a solution
to the heat equation. 

To see the bound, we compute: 
\begin{multline*} 
\norm{ \nabla u_\kappa(\cdot, t)} 
	= \norm{ \sqrt{\Delta_{g,\cW}} u_\kappa(\cdot, t)} 
	= \norm{ \sqrt{\Delta_{g,\cW}} e^{-t\Delta_{g,\cW}} e^{-\kappa \Delta_{g,\cW}}u_0} \\
	= \norm{ e^{-t \Delta_{g,\cW}} \sqrt{ \Delta_{g,\cW}} e^{-\kappa \Delta_{g,\cW}}u_0}
	\leq \norm{\sqrt{ \Delta_{g,\cW}} e^{-\kappa \Delta_{g,\cW}}u_0},
\end{multline*}
where the penultimate equality follows from the fact that
$\ran(e^{-\kappa \Delta_{g,\cW}}) \subset \dom(\Delta_{g,\cW}^\alpha)$
for all $\alpha > 0$ and via functional calculus.
The proof is complete on setting the constant 
$C(\kappa) = \norm{\sqrt{ \Delta_{g,\cW}} e^{-\kappa \Delta_{g,\cW}}u_0}$.
\end{proof}


As in \cite{CS}, for $T > 0$, define 
$$ W(T) = \set{ w \in \Lp{2}( (0,T); \SobH[0]{1}(V_x,g): \partial_t w \in \Lp{2}((0,T); \Lp{2}(V_x, g)},$$
and let $W_0(T) = \set{ w \in W: w(0) = w(T) = 0}.$

\begin{lemma}
The solution $u_\kappa \in \Lp{2}((0, \tau - \kappa); \SobH{1}(V_x, g))$
and is a weak solution in the following sense:
\begin{equation}
\label{eq:testfn2} 
\int_{0}^{\tau - \kappa} \inprod{a_x u_\kappa(t), \partial_t w(t)}_{\Lp{2}(V_x,\delta_x)}\ dt 
		= \int_{0}^{\tau - \kappa } \inprod{B_x \nabla u_\kappa(t), \nabla w(t)}_{\Lp{2}(V_x, \delta_x)}\ dt
\end{equation}
for all $w \in W_0(\tau - \kappa)$.
\end{lemma}
\begin{proof}
Fix $w \in W_0$ and from \eqref{eq:soln1}, we have that
$$\inprod{a_x \partial_t u_\kappa (t), w(t)}_{\Lp{2}(V_x, \delta_x)} 
		= \inprod{ B_x \nabla u_\kappa(t), \nabla v(t)}_{\Lp{2}(V_x, \delta_x)}$$
for every $t \in (0, \tau - \kappa)$.
A calculation similar to that in Lemma \ref{lem:shiftsol} 
yields that $\norm{ \partial_t u_\kappa} \leq C_\kappa'$
for all $t \in (0, \tau - \kappa)$
on noting that $\partial_t u_\kappa = \Delta_{g,\cW} u_\kappa$.
This, along with the bound in Lemma \ref{lem:shiftsol}
shows that  $u_\kappa \in \Lp{2}((0, \tau - \kappa); \SobH{1}(V_x, g))$
and that $\partial_t u_\kappa \in \Lp{2}((0,\tau - \kappa); \Lp{2}(V_x, g))$. 

Moreover, note that $W(\tau- \kappa) \subset \SobH{1}((0,\tau-\kappa); \Lp{2}(V_x,g))$
and therefore $\Ck{\infty}((0,\tau-\kappa); \Lp{2}(V_x,g) \intersect W(\tau - \kappa)$
is dense in $W$ in the Sobolev norm. Therefore on 
choosing such a $w$, as well as the fact that $w(0) = w(\tau - \kappa) = 0$, 
by the fundamental theorem of calculus in the Banach 
valued setting (c.f. \cite{}), we obtain that
$$\int_{0}^{\tau - \kappa} \inprod{ \partial_t(a_x u_\kappa(t)), w(t)}_{\Lp{2}(V_x,g)}\ dt
	= \int_0^{\tau - \kappa} \inprod{a_x u_\kappa(t), \partial_t w(t)}_{\Lp{2}(V_x, g)}\ dt.$$
Also, the integral on the left is equal to:
$$\int_{0}^{\tau - \kappa} \inprod{ \nabla u_\kappa(t), \nabla w(t)}\ dt$$
and by the aforementioned density of
 $\Ck{\infty}((0,\tau-\kappa); \Lp{2}(V_x,g) \intersect W(\tau - \kappa)$ 
in $W(\tau - \kappa)$ along with the fact that $\nabla u_\kappa \in \Lp{2}((0, \tau - \kappa); \Lp{2}(V_x,g))$,
the proof is complete.
\end{proof} 
\note{For this lemma, we need to find the right references
to the assertion about the density of this smooth Banach valued
functions as well as the f.t.c.}

With these three lemmas in hand, we prove
the main theorem. 

\begin{proof}[Proof of Theorem \ref{thm:Harnack}]
Let $\epsilon > 0$ be sufficiently small to 
be determined later. 
Fix $0 < \kappa < \tau < \infty$ and suppose that $u(x,t) \geq 0$ 
is a positive solution of
the $\Delta_{g,\cW}$ in $(\kappa,\tau)$. 
Then, as before, write $u_\kappa(x,t) = u(x,t+\kappa)$.

Let $D$ be any $n$-dimensional cube inside $\psi_x(V_x)$ then 
on writing $D_x = \psi_x^{1}(D)$, we have  
$$ \int_{D_x} a_x(y)\ d\Leb_x(y) \leq C_x^{\frac{n}{2}} \Leb_x(C),
\ \text{and}\ 
\int_{D_x} \frac{1}{a_x(y)}\ d\Leb_x(y) \leq C_x^{\frac{n}{2}}.$$
Therefore, our density $a_x$ satisfies the so called $A_2$ condition:
$$c_0 := \sup_{D_x} \cbrac{ \frac{1}{\Leb_x(D_x)} \int_{D_x} a_x(y)\ d\Leb_x(y)}
\cbrac{ \frac{1}{\Leb_x(D_x)} \int_{D_x} \frac{1}{a_x(y)}\ d\Leb_x(y)} 
	\leq C_x^n.$$
Combining this with the \eqref{eq:coeff} as
well as Lemma \eqref{eq:testfn2} shows that $u_\kappa$
satisfies the hypotheses of Theorem 2.1 in \cite{CS} and
therefore, we obtain a $\gamma = \gamma(C_x, n, \kappa, \tau) > 0$
such that 
\begin{equation} 
\label{eq:harnack2} 
\sup_{(y,s) \in Q_{(x,t-\kappa)}^-(0, \tau - \kappa, \epsilon)} u_\kappa(y,s)
		\leq \gamma \inf_{(y,s) \in Q^+_{(x,t-\kappa)}(0, \tau - \kappa, \epsilon)} u_\kappa(y,s).
\end{equation}
Note now that setting $s' = s + \kappa$, we have that
$(y, s' - \kappa) \in Q^{\pm}_{(x,t- \kappa)}(0, \tau - \kappa, \epsilon)$
if and only if $(y, s') \in Q^{\pm}(x,t)(\kappa, \tau, \epsilon)$.
That is, \eqref{eq:harnack2} is equivalent to the statement 
in the conclusion of the theorem and hence, this concludes
the proof.

Now, to compute a bound for $\epsilon$, note that
we want to ensure 
\begin{multline*}\set{(y,s) \in V_x \times (0, \infty): |y - s| < \epsilon, |t - s| < \epsilon^2} \\
\subset \set{(y,s) \in V_x \times (\kappa, \tau): |y - s| < \epsilon, |t - s| < \epsilon^2}
\end{multline*} 
we note that we require $\kappa < t - \frac{3}{4} \epsilon^2$,
$t + \epsilon^2 < \tau$ and $\epsilon < r_x$.
Rearranging this gives the range of $\epsilon$
in the conclusion.
\end{proof} 

\section{Regularity of solutions}

An important consequence of Harnack estimates for weak 
solutions is that they yield a priori regularity estimates
for those solutions. It is classical fact how these estimates
yield regularity results. For the benefit of the reader, 
we give the following brief outline of how to obtain
regularity from the estimates in Theorem \ref{thm:Harnack}.
We follow the argument of Theorem 6.28 in 
\cite{Lieberman} by Lieberman.
For the purposes of this section, let
$$ Q_{(x,t)}(\kappa,\tau, \epsilon) 	
	= \set{(y,s) \in V_x \times (\kappa, \tau): \modulus{x - y} \leq \epsilon, \modulus{t - s} < \epsilon^2}.$$ 

\begin{prop}
Let $u$ be a solution to the $\Delta_{g,\cW}$
heat equation (not necessarily positive).
Fix $x \in M$ and $0 < t_1 < t_2 < \infty$.
Then, there exists an open set $U_x \subset V_x$ 
with $x \in U_x$
such that 
$$ \modulus{u(y,s) - u(y',s)} \leq C(n, C_x, t_1, t_2) d_x(y, y')^\alpha,$$
where 
$$ \alpha = \log_{\frac{1}{4}}\cbrac{1 - \frac{1}{\gamma}}$$
for every $s \in [t_1, t_2]$.
\end{prop}
\begin{proof}
Let 
$\kappa = \frac{t_1}{2},\ \tau = \frac{3t_2}{2},\ R_0 = \frac{1}{4} \min\set{t_1, r_x},$ fix $r < \frac{R_0}{4}$.
Define, 
\begin{align*}
&M_4 = \sup_{(y,s) \in Q_{(x,t)}(\kappa, \tau, 4r)} u(y,s) &&m_4 = \inf_{(y,s) \in Q_{(x,t)}(\kappa, \tau, 4r)} u(y,s) \\
&M_1 = \sup_{(y,s) \in Q_{(x,t)}(\kappa, \tau, r)} u(y,s) &&m_1 = \inf_{(y,s) \in Q_{(x,t)}(\kappa, \tau, r)} u(y,s), 
\end{align*} 
and note that $M_j - u$ and $u - m_j$ for $j = 1, 4$ are non-negative
solutions to the $\Delta_{g,\cW}$ heat equation.
By the choice of $R_0$,
we always have that $r < \min\set{ \sqrt{t - \kappa}, \sqrt{\tau - t}, r_x}$
for every $t \in [t_1, t_2]$
and moreover,  $Q^{\pm}_{(x,t)}(\kappa,\tau,r) \subset Q_{(x,t)}(\kappa, \tau, r)$
So, fix such a $t \in [t_1, t_2]$ we invoke Theorem \ref{thm:Harnack} and integrate,
\begin{multline*}
\iint_{Q^-_{(x,t)}(\kappa,\tau,r)} (M_4 - u)\ d\Leb_x dt 
	\leq \gamma \iint_{Q^+_{(x,t)}(\kappa,\tau,r)} \inf_{Q^+_{(x,t)}(\kappa, \tau, r} (M_4 - u)\ d\Leb_x ds \\
	\leq \gamma (M_4 - M_1) \int_{t+ \frac{1}{4}r^2}^{t + r^2} \int_{\psi_x^{-1}(B(x,r))}\ d\Leb_x ds
	= \gamma w_n  \frac{1}{4}r^{2+n} (M_4 - M_1),$$
\end{multline*}
where $w_n$ is the constant for which $\Leb(B(x,r)) = w_n r^n$.
Note that the constant $\gamma$ is independent of $t$, and only 
dependent on $t_1$ and $t_2$ through our choice
of values for $\kappa$ and $\tau$. 
% Note:  $\inf (M_4 - u) = M_4 + \inf(-u) = M_4 - \sup{u} = M_4 - M_1$
Similarly, 
$$
\iint_{Q^-_{(x,t)}(\kappa,\tau,r)} (u - m_4)\ d\Leb_x dt \leq \gamma w_n \frac{3}{4}r^{2+n} (m_1 - m_4),$$
and adding these two inequalities together, 
we get that
$$
\frac{1}{2}r^{2+n} w_n (M_4 - m_4) \leq \gamma w_n \frac{1}{4}r^{2+n} (M_4 - m_4 +m_1 - M_1).$$
Rearranging, we find that 
$$
\osc_{Q_{(x,t)}(\kappa, \tau, r)} u \leq \cbrac{1 - \frac{1}{\gamma}} \osc_{Q_{(x,t)}(\kappa, \tau, 4r)} u.$$
We note that the constant $(1 - \gamma{-1}) < 1$
and therefore, it is of the right form to invoke the
standard iteration procedure given in Lemma 4.6 in \cite{Lieberman}. 
More specifically, on noting that
our oscillation estimate is of the form (4.15)'' in \cite{Lieberman},
we obtain the precise form for $\alpha$.

The passage from this to the Hölder estimate
we have noted in the conclusion is immediate. 
\end{proof}

\begin{rem}
The size of the parabolic cylinders
$Q^{\pm}_{(x,t)}(\kappa, \tau, r)$
are crucial for this estimate, since we need
the to ensure that the right hand constant 
can be made less than $1$. If the parabolic
region $Q^{-}_{(x,t)}(\kappa, \tau, r)$
needed to be much larger, then we could no 
longer organise this in the constant to 
obtain the regularity estimate in this way. 
\end{rem}

\printbibliography
\end{document}
