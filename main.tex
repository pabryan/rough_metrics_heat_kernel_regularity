\documentclass{amsart}

%\usepackage{etoolbox}
%\makeatletter
%\let\ams@starttoc\@starttoc
%\makeatother
%\makeatletter
%\let\@starttoc\ams@starttoc
%\patchcmd{\@starttoc}{\makeatletter}{\makeatletter\parskip\z@}{}{}
%\makeatother

%\usepackage[parfill]{parskip}
\usepackage{vmargin}
\usepackage[colorlinks=true,linkcolor=blue,citecolor=blue,urlcolor=blue]{hyperref}
\usepackage{bookmark}
\usepackage{amsthm,thmtools,amssymb,amsmath,amscd,amsfonts}
\usepackage{mathrsfs}
\usepackage{stmaryrd}


\usepackage[bibstyle=alphabetic,citestyle=alphabetic,backend=bibtex]{biblatex}
\bibliography{Bibliography}

\usepackage{fancyhdr}
\usepackage{esint}

\usepackage{enumerate}

\usepackage{pictexwd,dcpic}

\usepackage{graphicx}
\usepackage[utf8]{inputenc}

\declaretheorem[name=Theorem,numberwithin=section]{thm}
\declaretheorem[name=Remark,style=remark,sibling=thm]{rem}
\declaretheorem[name=Lemma,sibling=thm]{lemma}
\declaretheorem[name=Proposition,sibling=thm]{prop}
\declaretheorem[name=Definition,style=definition,sibling=thm]{defn}
\declaretheorem[name=Corollary,sibling=thm]{cor}
\declaretheorem[name=Assumption,style=remark,sibling=thm]{ass}
\declaretheorem[name=Example,style=remark,sibling=thm]{example}


\numberwithin{equation}{section}

\usepackage{cleveref}
\crefname{lemma}{Lemma}{Lemmata}
\crefname{prop}{Proposition}{Propositions}
\crefname{thm}{Theorem}{Theorems}
\crefname{cor}{Corollary}{Corollaries}
\crefname{defn}{Definition}{Definitions}
\crefname{example}{Example}{Examples}
\crefname{rem}{Remark}{Remarks}
\crefname{ass}{Assumption}{Assumptions}
\crefname{not}{Notation}{Notation}
\crefname{section}{Section}{Sections}

%Symbols
\renewcommand{\~}{\tilde}
\renewcommand{\-}{\bar}
\newcommand{\bs}{\backslash}
\newcommand{\cn}{\colon}
\newcommand{\sub}{\subset}

\newcommand{\N}{\mathbb{N}}
\newcommand{\R}{\mathbb{R}}
\newcommand{\Z}{\mathbb{Z}}
\renewcommand{\S}{\mathbb{S}}
\renewcommand{\H}{\mathbb{H}}
\newcommand{\C}{\mathbb{C}}
\newcommand{\K}{\mathbb{K}}
\newcommand{\Di}{\mathbb{D}}
\newcommand{\B}{\mathbb{B}}
\newcommand{\8}{\infty}

%Greek letters
\renewcommand{\a}{\alpha}
\renewcommand{\b}{\beta}
\newcommand{\g}{\gamma}
\renewcommand{\d}{\delta}
\newcommand{\e}{\epsilon}
\renewcommand{\k}{\kappa}
\renewcommand{\l}{\lambda}
\renewcommand{\o}{\omega}
\renewcommand{\t}{\theta}
\newcommand{\s}{\sigma}
\newcommand{\p}{\varphi}
\newcommand{\z}{\zeta}
\newcommand{\vt}{\vartheta}
\renewcommand{\O}{\Omega}
\newcommand{\D}{\Delta}
\newcommand{\G}{\Gamma}
\newcommand{\T}{\Theta}
\renewcommand{\L}{\Lambda}

%Mathcal Letters
\newcommand{\cL}{\mathcal{L}}
\newcommand{\cT}{\mathcal{T}}
\newcommand{\cA}{\mathcal{A}}
\newcommand{\cW}{\mathcal{W}}

%Mathematical operators
\newcommand{\INT}{\int_{\O}}
\newcommand{\DINT}{\int_{\d\O}}
\newcommand{\Int}{\int_{-\infty}^{\infty}}
\newcommand{\del}{\partial}

\newcommand{\inpr}[2]{\left\langle #1,#2 \right\rangle}
\newcommand{\fr}[2]{\frac{#1}{#2}}
\newcommand{\x}{\times}
\DeclareMathOperator{\Tr}{Tr}
\DeclareMathOperator{\Id}{Id}

\DeclareMathOperator{\dive}{div}
\DeclareMathOperator{\id}{id}
\DeclareMathOperator{\pr}{pr}
\DeclareMathOperator{\Diff}{Diff}
\DeclareMathOperator{\supp}{supp}
\DeclareMathOperator{\graph}{graph}
\DeclareMathOperator{\osc}{osc}
\DeclareMathOperator{\const}{const}
\DeclareMathOperator{\dist}{dist}
\DeclareMathOperator{\loc}{loc}
\DeclareMathOperator{\grad}{grad}
\DeclareMathOperator{\ric}{Ric}
\DeclareMathOperator{\Rm}{Rm}
\DeclareMathOperator{\weingarten}{\mathcal{W}}
\DeclareMathOperator{\inj}{inj}

%Environments
\newcommand{\Theo}[3]{\begin{#1}\label{#2} #3 \end{#1}}
\newcommand{\pf}[1]{\begin{proof} #1 \end{proof}}
\newcommand{\eq}[1]{\begin{equation}\begin{alignedat}{2} #1 \end{alignedat}\end{equation}}
\newcommand{\IntEq}[4]{#1&#2#3	 &\quad &\text{in}~#4,}
\newcommand{\BEq}[4]{#1&#2#3	 &\quad &\text{on}~#4}
\newcommand{\br}[1]{\left(#1\right)}

%Logical symbols
\newcommand{\Ra}{\Rightarrow}
\newcommand{\ra}{\rightarrow}
\newcommand{\hra}{\hookrightarrow}
\newcommand{\mt}{\mapsto}

%Names
\newcommand{\holder}{H\"older}

%Fonts
\newcommand{\mc}{\mathcal}
\renewcommand{\it}{\textit}
\newcommand{\mrm}{\mathrm}

%Spacing
\newcommand{\hp}{\hphantom}


%\parindent 0 pt

\protected\def\ignorethis#1\endignorethis{}
\let\endignorethis\relax
\def\TOCstop{\addtocontents{toc}{\ignorethis}}
\def\TOCstart{\addtocontents{toc}{\endignorethis}}


\newcommand{\note}[1]{\Rd {\bf[[ #1 ]]} \Bk}

\DeclareMathOperator{\met}{Met}
\DeclareMathOperator{\rough}{\mathcal{R}}
\DeclareMathOperator{\domain}{\mathcal{D}}
\newcommand{\ip}[2]{\ensuremath{\langle{#1},{#2}\rangle}}

\begin{document}

\title[Heat Kernel Regularity]
 {Heat Kernel Regularity For Rough Metrics}

\curraddr{}
\email{}
\date{\today}

\dedicatory{}
\subjclass[2010]{}
\keywords{}

\begin{abstract}
On smooth manifolds equipped with rough metrics, by employing a local Harnack inequality, we show there exists a heat kernel that is locally $C^{\alpha}$ regular.
\end{abstract}

\maketitle

\section{Introduction}
\label{sec:intro}

Let \(M\) be a smooth manifold. We consider \emph{rough metrics} on \(M\), the set of which we denote \(\met_{\rough} (M)\). These are metrics on \(M\) with measurable coefficients that are locally comparable to smooth smooth metrics (see \Cref{sec:rough_metrics}).

Our main theorem is the following:

\begin{thm}
On a smooth manifold \(M\), equipped with a rough metric \(g\), there exists a heat kernel \(\rho_t = \rho_t(g)\), such that
\begin{enumerate}
\item \(\rho_t > 0\) for \(t > 0\),
\item on a parabolic cylinder \(Q = K \times [t_1, t_2]\) with \(0 < t_1 < t_2\), \(K \subseteq M\) is compact, there exists an \(\alpha = \alpha(Q)\) so that \(\rho_t \in C^{\alpha}(K \times K)\) for \(t_1 \leq t \leq t_2\).
\end{enumerate}
\end{thm}

\section{Rough Metrics}
\label{sec:rough_metrics}

\begin{defn}
Let \(M\) be a smooth manifold. A set \(E \subseteq M\) is said to be \emph{Lebesgue measurable} if for all charts \(\varphi : U \subseteq M \to V \subseteq \R^n\), the set \(\varphi(E \cap U)\) is a Lebesgue measurable set of \(\R^n\).
\end{defn}

\begin{rem}
The Lebesgue measurable sets form a \(\sigma\)-algebra. The notion of Lebesgue measurable set does not involve a measure on \(M\). However, if \(\omega\) is volume form on \(M\), then it defines a (outer) measure \(E \mapsto \int_E \omega\) for which the \(\sigma\)-algebra of measurable sets is precisely the \(\sigma\)-algebra of Lebesgue measurable sets since with respect to any chart, \(\omega\) is absolutely continuous with respect to Lebesgue measure on \(\R^n\).
\end{rem}

\begin{defn}
Let \(M, N\) be smooth manifolds. A function \(f : M \to N\) is \emph{Borel measurable} if for all Borel sets \(E \subset N\), \(f^{-1} (E)\) is a Borel set of \(M\). A function \(f : M \to N\) is \emph{Lebesgue measurable} if for all Lebesgue measurable sets \(E \subset N\), \(f^{-1} (E)\) is a Lebesgue measurable set of \(M\).
\end{defn}

\begin{rem}
Equivalently, \(f\) is Lebesgue measurable if for all charts \(\phi : U \subseteq M \to V \subseteq \R^n\) and \(\psi: W \subseteq N \to Z \subseteq \R^m\), the function \(\psi \circ f \circ \phi^{-1} : \phi \circ f^{-1} (W) \subseteq \R^n \to Z \subseteq \R^m\) is Lebesgue measurable.
\end{rem}

\textbf{What about \(f^{-1}(E)\) is Lebesgue measurable for every Borel set \(E \subseteq N\)?}

\begin{defn}
A \emph{rough} metric \(g\) is a Lebesgue measurable section of \(\met(M) = (T^{\ast} M \odot T^{\ast} M)_+\), the bundle of positive definite, symmetric bilinear forms on \(TM\) that is \emph{locally comparable} to smooth metrics. That is, \(g : M \to \met(M)\) is a Lebesgue measurable function such that \(\pi \circ g = \id_M\) where \(\pi: \met(M) \to M\) is the bundle projection. To say that \(g\) is locally comparable to smooth metrics is to say that there is an open cover \(\lbrace U_{\alpha} \rbrace\) of \(M\), smooth metrics \(g_{\alpha} \in \met(U_{\alpha})\) and real constants \(C_{\alpha} > 0\) such that
\[
\frac{1}{C_{\alpha}} g_{\alpha} (X, Y) \leq g(X, Y) \leq C_{\alpha} g_{\alpha} (X, Y)
\]
for all tangent vectors \(X, Y \in TM\).
\end{defn}

\begin{rem}
By employing a partition of unity argument, we can patch together the metrics \(g_{\alpha}\) to produce a globally defined metric \(h\). If the constants \(C_{\alpha}\) are uniformly bounded above and away from zero, then our rough metric will be globally comparable to \(h\). This is automatic whenever \(M\) is compact. If \(M\) is not compact, then this need not be true. Previously a Harnack inequality and heat kernel estimates have been obtained when the metric is globally comparable to a complete metric with bounded below curvature \cite{Saloff-Coste:/1992b}. Here we do not require the assumptions of global comparability.
\end{rem}

Given a rough metric, we may locally define a \(g_{\alpha}\)-self-adjoint, measurable operator \(A_{\alpha} : U_{\alpha} \to T^1_1 U_{\alpha}\) by
\[
g(X, Y) = g_{\alpha} (A_{\alpha} (X), Y).
\]

\section{The Laplacian of a Rough Metric}

For a smooth function \(f \in \C^{\infty}(M)\), write \(\nabla f = df\) for the differential of \(f\). The differential \(\nabla : C^{\infty}_0(M) \to \Gamma^{\infty}_0(T^{\ast} M)\), maps smooth, compactly supported functions \(C^{\infty}_0(M)\) to smooth compactly supported one-forms \(\Gamma^{\infty}_0(T^{\ast} M)\).

As an unbounded operator \(\nabla: L^2_g (M) \to \Gamma^2_g(T^{\ast} M)\) from \(L^2\) functions \(L^2_g (M)\) to \(L^2\) one-forms \(\Gamma^2_g(T^{\ast} M)\), \(\nabla\) is a closable operator. The closure \(\bar{\nabla}\) of \(\nabla\) has domain
\[
\domain(\bar{\nabla}) = W^{1,2}_g (M) = \lbrace \varphi \in L^2 (T^{\ast} M) : \nabla \varphi \in \Gamma^2_g(T^{\ast} M) \rbrace
\]
the Sobolev space of \(L^2\) functions with \(L^2\) differential.

We write \(\nabla^{\ast}_g : \Gamma^2_g(T^{\ast} M) \to \Gamma^2_g(M)\) for its adjoint which satisfies
\[
\ip{\nabla f}{X}_{g} = \int_M g(\nabla f, X) d\mu = \int_M f \nabla^{\ast}_g X d\mu_g = \ip{f}{\nabla^{\ast}_g X}_g
\]
for \(f \in \domain(\nabla)\) and \(X \in \domain(\nabla^{\ast}_g)\) where \(\domain(T)\) denotes the domain of an (unbounded) operator \(T\).

\begin{rem}
Let us remark that \(\nabla\) is defined purely from the smooth structure on \(M\) independently of any metric (rough or otherwise) on \(M\). On the other hand, \(\nabla^{\ast}_g\) depends on the smooth structure and the (rough) metric \(g\).
\end{rem}

\begin{defn}
Let \(g\) be a rough metric on a smooth manifold \(M\). The \emph{Laplacian} \(\Delta_g : L^2_g(M) \to L^2_g(M)\) is defined by
\[
\Delta_g f = \nabla^{\ast}_g \nabla f.
\]
\end{defn}

\begin{rem}
The Laplacian is (by definition) \(L^2_g\) self-adjoint:
\[
\ip{\Delta_g f}{h}_g = \int_M (\nabla^{\ast}_g \nabla f) h d\mu_g = \int_M g(\nabla f, \nabla h) d\mu_g = \int_M f \nabla^{\ast}_g \nabla h d\mu_g = \ip{f}{\Delta h}_g.
\]
Therefore, the heat semi-group
\[
e^{t \Delta} : L^2 \to L^2
\]
is defined for all \(t > 0\).
\end{rem}

\begin{defn}
Let \(I \subset \R\) be an interval. A function \(u \in C^1(I \to W^{1,2}(M))\) solves the \emph{heat equation} if for each \(t \in I\),
\[
\partial_t u (\cdot, t) = \Delta u (\cdot, t)
\]
in \(W^{1,2}(M)\).
\end{defn}

\begin{rem}
For any solution \(u\), of the heat equation, we have the representation formula
\[
u(x, t) = \left(e^{-t \Delta} u(\cdot, 0)\right) (x).
\]
\end{rem}

\section{The Harnack Inequality and Heat Kernel Regularity}
\label{sec:harnack}


\end{document}
